\documentclass[11pt]{article}
\usepackage{times}
    \usepackage{fullpage}
    
    \title{Literature Review}
    \author{Mok Kah Hou - 2689719M}

    \begin{document}
    \maketitle
    
\section{Introduction}
    \subsection{Context}
    % Briefly explain what visual field mapping is and why it's important, especially in the context of VR.
    Visual field testing is to chart the visual field of a patient, to find blind spots and blurry spots of vision. Can use that data to construct a visual field map which is in the brain, and from there diagnose possible defects in the visual pathway or even in the visual cortex of the brain.
    
    But VFT have not changed much, and with upcoming VR headset technology which mainly focus on the eyes, can create a easy to distribute, and easy to use but still relatively accurate VFT using VR.
    \subsection{Objective}
    Purpose of report is to, identify current visual field technologies, and subsequently research gaps in said subjects
    
    Find trends and methodologies other researchers have used, and plan to implement them in my project.
    
    % State the purpose of report, such as identifying trends, key methodologies, and research gaps in visual field mapping using VR.
    \subsection{Scope}
    % Mention the criteria for paper selection and briefly explain triage process (e.g., focusing on abstracts, introductions, and conclusions).
    Criteria of paper selections into scope: Keywords in titles, References by past works on similar project work. 
    

\section{Methodology}
    \subsection{Selection Criteria and Triage}
     % Explain how I chose the papers (e.g., keywords, relevance to VR and visual field mapping).
     Start of by searching on google scholar keywords such as Visual Field Testing, VR etc, and reading the titles of the papers, Only clicking in on those that shows relevancy to my subject. Then read through the abstract further refine the choices of papers. After reading through the abstract, take note of those that have higher relevance or those which shows promising info on my subject. These will be selected to be read through more in depth, such as the introductions and discussions and conclusions

    Also take notes of research papers cited by papers that I have read and deemed significant to my subject.
    % \subsection{Triage}
    % Briefly describe my approach, such as reading abstracts for relevance and then delving into details if the paper fits.

\newpage
\section{Review and Analysis of Selected Literature}
\subsection{Current Technology on Visual Field Testing}
    Overview: All these papers bring reader up to date on the development of visual field testing and why they are important
        \subsubsection{A History of Perimetry and Visual Field Testing}
            Author: Chris A. Johnson*
            Year: 2011
            \begin{itemize}
                \item Objective:  Provide a brief historical overview of VFT procedures
                \item Methodology: -
                \item Key Findings: VFTs have not change by a lot, Different types of VFTs that may be useful to my project.
                \item Relevance:  Provides basic understanding on the different types of VFTs and the thought process to construct them, such as pros and cons, aid in choosing which VFT to use.
            \end{itemize}

        \subsubsection{Visual Field Tests: A Narrative Review of Different                  Perimetric Methods}
            Author: Bhim Bahadur Rai*
            Year: 2024
            \begin{itemize}
                \item Objective: Summarizes each of the different VF tests and perimetric methods, and why VFTs are important
                \item Methodology: Researched online for different VF testing and perimetric methods
                \item Key Findings: Usefulness of VFTs, electroretinograms are inconvenient
                \item Relevance: Provides more understanding on VFTs, such such as pros and cons, aid in choosing which VFT to use.
            \end{itemize}

        \subsubsection{Reliability of Visual Field Results over Repeated                    Testing}
            Author: Joanne Katz*
            Year: 1991  
            \begin{itemize}
                \item Objective: Examine whether the reliability of automated perimetry improves with yearly testing.
                \item Methodology: Conducted study on self-referred subjects at yearly intervals.
                \item Key Findings: Found that 35\% of patients
                with ocular hypertension had unreliable results on initial testing and that the proportion declined to 26\%.
                \item Relevance: To support the basis of easy repeatable testing that comes with a deployed accessible VR application.
            \end{itemize}

\newpage
\subsection{Theme 2}
    Overview: 
        \subsubsection{Title, Author, Year}
            \begin{itemize}
                \item Objective:
                \item Methodology:
                \item Key Findings:
                \item Relevance:
            \end{itemize}

        \subsubsection{Title, Author, Year}
            \begin{itemize}
                \item Objective:
                \item Methodology:
                \item Key Findings:
                \item Relevance:
            \end{itemize}

        \subsubsection{Title, Author, Year}
            \begin{itemize}
                \item Objective:
                \item Methodology:
                \item Key Findings:
                \item Relevance:
            \end{itemize}

\newpage
\subsection{Theme 3}
    Overview:
        \subsubsection{Title, Author, Year}
            \begin{itemize}
                \item Objective:
                \item Methodology:
                \item Key Findings:
                \item Relevance:
            \end{itemize}

        \subsubsection{Title, Author, Year}
            \begin{itemize}
                \item Objective:
                \item Methodology:
                \item Key Findings:
                \item Relevance:
            \end{itemize}

        \subsubsection{Title, Author, Year}
            \begin{itemize}
                \item Objective:
                \item Methodology:
                \item Key Findings:
                \item Relevance:
            \end{itemize}

\newpage
\section{Comparative Analysis}
    \subsection{Key Trends}
     Summarize trends or patterns across the papers, such as common VR methods, success in visual field accuracy, or user interaction insights.
    \subsection{Gaps in Research}
     Identify areas that are under-explored or need further investigation, such as specific challenges in VR simulation accuracy for visual field mapping.

\section{Conclusion}
    \subsection{Summary of Insights}
     Briefly highlight the most important findings that are relevant to my project.
    \subsection{Implications for Project}
     Explain how these insights will inform or guide my own work.
    \subsection{Future Directions}
     Suggest potential research directions that could benefit the field and help advance my project.

\newpage
\section{References}

% \section{Title 1}
%     \subsection{Author, Date}
%         [Author, Date]
%     \subsection{Methodology}
%         Briefly describe the approach or methodology, especially if it relates to VR or visual field mapping techniques relevant to project.
%     \subsection{Key findings}
%         Note the main findings and conclusions. Focus on aspects like the effectiveness of VR in visual field mapping, specific techniques used, or results related to field coverage, accuracy, or user experience.
%     \subsection{Relevance to Project}
%         Write a few sentences on why this paper is or isn’t relevant. This will help shape my report and justify the papers included.

\end{document}